%!TEX root=../thesis.tex
\section{Computer Networks}
The following sections give a short introduction into computer network theory and provide an explanation to the terms used in this thesis.
If not otherwise noticed, the information of this section is taken from~\cite{tanenbaum2011networks}.

\begin{figure}
  \centering
  \def\svgwidth{\columnwidth}
  \input{./figures/osi.pdf_tex}
  \caption[\acrshort{OSI} Model]{The first 4 layers of the \gls{OSI} model. Host 1 sends a message to Host 2, one layer 3 and one layer 2 switch are involved.}
  \label{fig:osi}
\end{figure}

\paragraph{OSI model}
The \gls{OSI} model is a reference model for computer network architecture proposed by the \gls{ISO} in~\cite{isoosi1994}.

It defines seven layers which are stacked on top of each other like displayed in figure~\ref{fig:osi}.
Every layer is just able to communicate with its neighbouring layers and performs a well defined function, such as layer 1 (physical layer) transmits bytes on a medium.

When an application transmits data to another host the data is traveling the path from the top layer (application layer) to the bottom layer, which is the physical layer.
Each layer on this path performs its function on the data.
The recipient receives the data and each layer processes it, from the lowest back to the top layer.
Also the devices in between the network have a layered architecture and are processing the transmitted data in their layers.
Those devices in the network which determine the path of transmitted data (``routing'') usually have three or two layers as shown in figure~\ref{fig:osi}.

Data received by a layer from the layer above is named \gls{SDU}.
The processed data, which a layer is passing down to the next layer, is called \gls{PDU}.
\glspl{PDU} have different names, depending on the layer.
On layer 2 they are named ``frames'' and on layer 3 they are named ``packets''.
 A generic term is ``datagram''~\cite{RFC1594}.

Relevant layers for this thesis are the network layer (layer 3) and the data link layer (layer 2).

\paragraph{Layer 3}
\label{par:layer-3}
The network layer is determining the route of a packet and makes communication with other networks possible.
Furthermore, it adapts the \gls{SDU} to the \gls{MTU}, which is the maximum size a data packet can have.
Therefore, it breaks the \gls{SDU} in several fragments and sends them as single packets.
The receiving network layer is able to reassemble the fragments and passes them as a whole to the overlaying layer.
This is possible because each layer is able to put more information into each \gls{SDU} by transforming it or prepending and appending data.
Usually a layer prepends a header with information on the \gls{SDU} on transmission.
When it receives a \gls{PDU} to pass it to an upper layer, the information is processed and stripped.
One of the usual protocols that work on layer 3 is the \gls{IP}.
The fragmentation process of \gls{IP} is described in section~\ref{sec:ip-fragmentation}.

\paragraph{Layer 2}
The data link layer communicates in the same subnetwork and consists of two sublayers.
The Logical Link Control sublayer enables multiplexing.
On shared channels the \gls{MAC} sublayer is checking if the channel is already in use and if a frame can be transmitted.
Furthermore, it passes data to the physical layer.
The \gls{SDU} is put into a frame, source and destination addresses are added, as well as a checksum for error detection.
In a cable-based \gls{LAN} usually Ethernet is used at the data link layer.
A \gls{LAN} is a network which operates in a small area, such as one building.

\begin{figure}
  \centering
  \begin{bytefield}[bitwidth=0.04\columnwidth]{25}
    \bitbox{4}{Preamble} & \bitbox{4}{Destination} & \bitbox{4}{Source} &
    \bitbox{2}{Type} &
    \bitbox{6}{Payload} &
    \bitbox{1}{\tiny P\\a\\d} & \bitbox{4}{Checksum}
  \end{bytefield}
  \caption{Ethernet \acrlong{PDU}}
  \label{fig:ethernet}
\end{figure}

\paragraph{Switched Ethernet}
Ethernet is standardized by the \gls{IEEE} 802.3\footnote{\url{http://www.ieee802.org/3/}} working group.
The terms ``\gls{IEEE} 802.3'' and ``Ethernet'' are interchangeable.
Classical Ethernet used a shared medium to transmit data, which means the physical cable was used by more than one participant at a time.
This evolved to Switched Ethernet, here a dedicated cable per transmission channel is used.

Data is sent in frames and each frame has a preamble (7 bytes), source (6 byte), destination (6 byte) and ethertype (2 byte) prepended as shown in figure~\ref{fig:ethernet}.
The preamble is a constant value to synchronize the clocks of sender and receiver.
Source and destination addresses are 6 bytes long and unique, so every participant can derive if he is the receiver of the frame.
The ethertype field (2 bytes) is used to distinguish different protocols, so the receiver knows what type of data the payload is.
Ethertypes can be requested by anyone at the \gls{IEEE}.
The Ethertype for \gls{MACsec} is 0x88E5.
Furthermore, a checksum (4 bytes) is appended, so the receiver is able to detect errors that happend on transmission.
If an error is detected the frame is dropped, so Ethernet does not guarantee a successful transmission.
An Ethernet frame has to have a minimum size of 64 bytes and a maximum size of 1522 bytes without the preamble, therefore, the payload can usually have a maximum size of 1500 bytes.
If it has less than 46 bytes, padding has to be used.

\paragraph{Switches \& Router}
A switch is a device which has a direct connection to several participants or other switches and forwards received frames to the destination.
A layer 2 switch uses the information in the Ethernet frame header to determine the destination.
As it works on layer 2, it can just forward packets in the same network.
A layer 3 switch---or router---is able to route packets and uses the information of the \gls{IP} to determine the destination.
Therefore, it is able to route between different networks or fragment packets.

\paragraph{MTU}
The \acrfull{MTU} is the maximum size of a data unit to be transmitted.
As prior mentioned, the Ethernet \gls{MTU} is for example 1500 bytes, for \acrlong{WLAN} it is 2272 bytes~\cite{ieeewlan}.
The protocol used in layer 3---which is in practice mostly \gls{IP}---fragments the \glspl{SDU} to match the \gls{MTU}.

The Path \gls{MTU} is the minimum \gls{MTU} on the path a packet travels.
To determine the Path \gls{MTU} a process named \gls{PMTUD} is used, which is specified in \cite{RFC1191}.

\paragraph{Jumbo Frames}
Additionally, there is the possibility of raising the \gls{MTU} up to 9000 bytes while using at least Gigabit Ethernet.
Frames with a size higher than 1500 bytes are called Jumbo Frames.
This is not standardized, but supported by most vendors of network interfaces.

\paragraph{\acrlong{TCP}}
\gls{TCP} is a protocol which resides on layer 4---the transport layer---of the \gls{OSI} model.
It was first specified in RFC~793~\cite{RFC0793}.
It builds a connection over a network and provides a reliable transmission of packets.
To achieve reliability, the receiver sends an acknowledgement for received packets.
If these are not received by the sender after a certain time the packets are resend.

\paragraph{\acrlong{ICMP}}
The \gls{ICMP} is a companion protocol for \gls{IP}.
Its first specification was in~\cite{RFC0792}.
If for example the size of an \gls{IP} packet is bigger than the \gls{MTU}, an \gls{ICMP} message is sent to report this to the sender.
Furthermore, it is used to send ``echo'' messages to check if a machine is reachable which then answers with an ``echo reply'' message.
This can be used to measure the \gls{RTT}, which is the time from sending a packet until receiving the acknowledgement.
