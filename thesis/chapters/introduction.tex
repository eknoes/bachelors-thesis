%!TEX root=../thesis.tex
\chapter{Introduction}
\label{ch:intro}
As Industry 4.0 becomes more popular, networking in factories has gotten more important.
However, current factories are insecure as a study of the german Federal Ministry of Economic Affairs and Energy regarding IT security in Industry 4.0 shows~\cite{studiebmwi}.
The FastVPN project\footnote{\url{https://en.fast-zwanzig20.de/industrie/fast-vpn-2/}} adresses this problem space.

It provides transparent network bridges---FastVPN boxes---, which are deployed between industry machines and the factory network.
Thus the network communication is encapsulated, furthermore, network security inbetween the FastVPN boxes is improved.
This environment is displayed in figure~\ref{fig:environment}.

\begin{figure}[!h]
  \centering
  \def\svgwidth{0.8\columnwidth}
  \input{./figures/environment.pdf_tex}
  \caption[FastVPN Setup Example]{Insecure network communication is protected by FastVPN boxes. The is done by applying MACsec. The \acrlong{MTU} between the boxes is smaller than the one between client and box.}
  \label{fig:environment}
\end{figure}

A protocol used by the FastVPN box to improve network security is \gls{MACsec}, a standard of the \gls{IEEE}.
\gls{MACsec} provides confidentiality and integrity on the data link layer.
It encrypts network communication and protects against unauthorized modification by appending a message authentication code.
The downside of \gls{MACsec} is the reduction of the maximum packet size---the \gls{MTU}--- because additional information are added to each frame which is tunneled through the FastVPN box.

If machines cannot discover that the \gls{MTU} is decreased on the path, the packets are oversized at one point.
The standard behavior is to drop these oversized packets.
As a result, machines without this discovery mechanism are not able to communicate through the FastVPN box.

A solution to this problem can be the use of jumbo frames, which allow higher \gls{MTU} sizes than the standard of 1500 bytes.
By enabling jumbo frames between the FastVPN boxes, the \gls{MTU} can be increased so that the overhead introduced by \gls{MACsec} is equalized.
But to enable jumbo frames modern network hardware between the FastVPN boxes is required.

The aim of the FastVPN project is the deployment in factories.
Machines as well as network hardware in factories often have a long life span, so the FastVPN project cannot assume that the configuration of jumbo frames is feasible.
Furthermore, it is not assumed that the machines are discovering the decreased \gls{MTU}.

Therefore, it requires a solution to use \gls{MACsec} without a reduction of the \gls{MTU} and the configuration of jumbo frames.
The aim of this thesis is to develop a transparent fragmentation solution in combination with \gls{MACsec} to solve this problem.
As the reduction of the \gls{MTU} is comparatively small to the frame size, a simple split produces one full and one very small frame.
Hence, possible optimizations are examined.

This thesis begins with an introduction into computer networks, security and \gls{MACsec}.
Moreover, protocols which support fragmentation and possible optimizations are examined.
Afterwards, several approaches to implement fragmentation when using \gls{MACsec} are proposed and discussed.
The most promising solution is then specified in detail and implemented for the use with linux.
Finally, the implementation of the proposed fragmentation scheme is evaluated regarding security and performance.
Therefore, it is compared with the solution of using jumbo frames.
