%!TEX root=../thesis.tex
\chapter{Proposed Solution}
\label{ch:proposal}
Fragmenting \glspl{SDU} solves the \gls{MACsec} \gls{MTU} problem and is superior compared to \gls{MPDU} fragmentation as discussed in the previous chapter.
Therefore, this chapter specifies the proposed modifications of \gls{MACsec} to realize transparent \gls{SDU} fragmentation and concatenation.
Afterwards, the implementation is described.

\section{Specification}
\begin{figure}
    \centering
    \resizebox{\columnwidth}{!}{
      %!TEX root=../thesis.tex
%\documentclass{article}
%\usepackage{geometry}
%\usepackage{tikz}
%\usetikzlibrary{matrix,calc,shapes,positioning}
%\begin{document}
  \begin{tikzpicture}
    [node distance=5.5cm,
    startstop/.style={rectangle, minimum width=3cm, minimum height=1cm,text centered, draw=black, fill=pastelred},
    decision/.style={diamond, minimum width=4cm, minimum height=4cm, text centered, draw=black, fill=pastelgreen},
    process/.style={rectangle, minimum width=3cm, minimum height=1cm, text centered, draw=black, fill=pastelorange, inner sep=.2cm}]
    \node (start) [circle,fill=black,minimum size=.5cm,label=above:SDU in queue] {};
    %\node (start) [startstop, below=1cm of begin] {SDU should be sent};
    %\draw[-{Latex[width=3mm]}] (begin) -- (start);
    \node (frame-size) [decision, right=4cm of start,align=center] {Compare frame\\size to MTU};
    \draw[-{Latex[width=3mm]}] (start) -- (frame-size);

    \node (concat-enabled) [decision, below=1cm of frame-size, align=center] {Concatenation\\ enabled?};
    \draw[-{Latex[width=3mm]}] (frame-size) -- node[anchor=east,align=center,yshift=.1cm] {Undercuts\\MTU} (concat-enabled);

    \node (sdu-queue) [decision, below=1cm of concat-enabled, align=center] {SDU with\\same SC\\enqueued?};
    \draw[-{Latex[width=3mm]}] (concat-enabled) -- node[anchor=east,yshift=.3cm]{true} (sdu-queue);

    \node (dequeue) [process, below=1cm of sdu-queue] {Dequeue SDU};
    \draw[-{Latex[width=3mm]}] (sdu-queue) -- node[anchor=east,yshift=.3cm]{true} (dequeue);
    \node (cut-source) [process, left=1.5cm of dequeue, align=center] {Strip source\\ and destination\\of SDU};
    \draw[-{Latex[width=3mm]}] (dequeue) -- (cut-source);
    \node (append) [process, above=1cm of cut-source, align=center] {Append SDU as new\\segment to payload};
    \draw[-{Latex[width=3mm]}] (cut-source) -- (append);
    \node (save-length) [process, above=1cm of append, align=center] {Save\\ segment length};
    \draw[-{Latex[width=3mm]}] (append) -- (save-length);
    \draw[-{Latex[width=3mm]}] (save-length) |- (frame-size);

    \node (split-sdu) [process, right=5cm of frame-size, align=center,yshift=3cm] {Trim frame to MTU\\length, save remainder\\as second fragment};
    \draw[-{Latex[width=3mm]}] (frame-size) |- node[anchor=east, align=center,yshift=-.5cm]{Exceeds\\MTU} (split-sdu);
    \node (copy-src) [process, below=1cm of split-sdu, align=center] {Copy source and destination\\to second fragment};
    \node (enqueue-fragment) [process, below=1cm of copy-src, align=center] {Enqueue second\\fragment as\\next SDU};
    \draw[-{Latex[width=3mm]}] (split-sdu) -- (copy-src);
    \draw[-{Latex[width=3mm]}] (copy-src) -- (enqueue-fragment);
    \node (save-mf) [process, below=1cm of enqueue-fragment, align=center] {Save More-Fragments bit\\for SecTAG creation};
    \node (end) [startstop, right of=sdu-queue] {Apply MACsec};

    \node (stop) [circle,minimum size=.5cm, line width=.01cm, fill=black, below=1cm of end] {};
    \node (stop2) at (stop) [circle,minimum size=.7cm, line width=.01cm, draw=black,label=below:Transmit MPDU] {};

    \draw[-{Latex[width=3mm]}] (frame-size) -| node[anchor=south,pos=.0,xshift=1.2cm] {Matches MTU} (end);
    \draw[-{Latex[width=3mm]}] (concat-enabled) -| node[anchor=south,pos=.0,xshift=0.4cm]{false} (end);
    \draw[-{Latex[width=3mm]}] (sdu-queue) -- node[anchor=south,pos=.0,xshift=0.4cm]{false} (end);

    \draw[-{Latex[width=3mm]}] (enqueue-fragment) -- (save-mf);
    \draw[-{Latex[width=3mm]}] (save-mf) |- (end);
    \draw[-{Latex[width=3mm]}] (end) -- (stop2);

    \begin{pgfonlayer}{bg}
      \node at (save-length) [draw,color=white,fill=gray!20,rectangle,yshift=-1.5cm,xshift=2cm,minimum width=11cm,minimum height=9.5cm,label={[xshift=-4cm,align=left]above:\textsc{Concatenation}}] {};

      \node at (copy-src) [draw,color=white,fill=gray!50,rectangle,xshift=0cm,yshift=-1cm,minimum width=6cm,minimum height=10cm,label={[xshift=-1.5cm]above:\textsc{Fragmentation}}] {};
    \end{pgfonlayer}

  \end{tikzpicture}
%\end{document}

    }
    \caption[Transmission Flowchart]{Transmission flowchart for proposed modifications}
    \label{fig:fragmentation-flowchart}
\end{figure}
\subsection{Transmission Process}

The following transmission flow is displayed as flowchart in figure~\ref{fig:fragmentation-flowchart} for a better understandability.

When a frame has to be sent using \gls{MACsec} it must be checked at the beginning whether fragmentation has to be applied.
This is the case if the size of the \gls{SDU} exceeds the maximum payload size of \gls{MACsec} regarding the \gls{MTU} of the underneath layer.
If the standard \gls{MTU} of 1500 bytes is set, the maximum payload size is 1468 bytes.

The fragmentation process must happen before applying \gls{MACsec}.
Therefore, the \gls{SDU} must be split.
The first fragment must be sent with the \acrfull{MF} bit set.
It must be ensured that the second fragment is processed next by \gls{MACsec}.

If the size of the \gls{SDU} undercuts the maximum payload size of \gls{MACsec}, concatenation must be applied if enabled.
Therefore, the queue of \gls{MACsec} has to be checked for additional \glspl{SDU} that belong to the same \acrfull{SC}.
If it contains \glspl{SDU}, they are dequeued and appended to the payload until the maximum payload size is matched or exceeded.
Source and Destination addresses of these \glspl{SDU} are cut.
It must be considered that for each additional \gls{SDU} the maximum payload is decreased by two bytes because an extension header must be added.
The last \gls{SDU} may be fragmented if otherwise the maximum payload size is exceeded.
The lengths of the additional \glspl{SDU} are saved for the creation of the \gls{SecTAG}.

When creating the \gls{SecTAG} for each additional \gls{SDU} an extension header is appended to the \gls{SecTAG}.
The extension header contains the size of the corresponding \gls{SDU}.
The $n$th extension header corresponds to the $n+1$th \gls{SDU}.
The \acrlong{E} bit is set for all but the last extension headers.

After the fragmentation and concatenation process, \gls{MACsec} is applied as usual.
\begin{figure}
    \centering
    \resizebox{\columnwidth}{!}{
      %!TEX root=../thesis.tex
%\documentclass{article}
%\usepackage{geometry}
%\usepackage{tikz}
%\usetikzlibrary{matrix,calc,shapes,positioning}
%\begin{document}
\begin{tikzpicture}
  [node distance=5.5cm,
  startstop/.style={rectangle, minimum width=3cm, minimum height=1cm,text centered, draw=black, fill=pastelred},
  decision/.style={diamond, minimum width=4cm, minimum height=4cm, text centered, draw=black, fill=pastelgreen},
  process/.style={rectangle, minimum width=3cm, minimum height=1cm, text centered, draw=black, fill=pastelorange, inner sep=.2cm}]
  \node (begin) [circle,minimum size=.5cm, line width=.01cm, fill=black,label=above:Receive MPDU] {};
  \node (macsec) [startstop, below=1cm of begin] {MACsec validation};
  \draw[-{Latex[width=3mm]}] (begin) -- (macsec);
  \node (e-bit) [decision, below=1cm of macsec] {E bit set?};
  \draw[-{Latex[width=3mm]}] (macsec) -- (e-bit);

% De-concatenate
\node (read-ext-header) [process, right=1.5cm of e-bit, align=center] {Read next\\extension header};
\draw[-{Latex[width=3mm]}] (e-bit) --node[anchor=south]{true} (read-ext-header);
\node (save-len) [process, below=2cm of read-ext-header] {Save segment length};
\draw[-{Latex[width=3mm]}] (read-ext-header) -- (save-len);
\draw[-{Latex[width=3mm]}] (save-len) -| (e-bit);
\node (split) [process, left=1.5cm of e-bit, align=center] {Split according to\\segment lengths};
\draw[-{Latex[width=3mm]}] (e-bit) --node[anchor=south]{false} (split);

% Reassembly
\node (buffer-empty) [decision, below=1cm of split, align=center] {Buffer contains\\segments?};
\draw[-{Latex[width=3mm]}] (split) -- (buffer-empty);
\node (append-to-buffer) [process, left of=buffer-empty, align=center] {Append first\\segment to buffer};
\draw[-{Latex[width=3mm]}] (buffer-empty) --node[anchor=south]{true} (append-to-buffer);
\node (deliver-buffer) [process, below=1cm of append-to-buffer,align=center] {Deliver buffered\\segment};
\draw[-{Latex[width=3mm]}] (append-to-buffer) -- (deliver-buffer);
\node (copy-src-dest) [process, below=2cm of buffer-empty,align=center] {Copy source and\\destination to each segment};
\draw[-{Latex[width=3mm]}] (deliver-buffer) |- (copy-src-dest);
\draw[-{Latex[width=3mm]}] (buffer-empty) --node[anchor=east,pos=.0,yshift=-.3cm]{false} (copy-src-dest);

% Fragmentation
\node (mf-set) [decision, below=1cm of copy-src-dest, align=center] {MF bit set?};
\node (buffer-last) [process, left of=mf-set,align=center] {Move last segment\\to buffer};
\node (deliver) [startstop, below=2cm of mf-set] {Deliver remaining segments};
\node (stop) [circle,minimum size=.5cm, line width=.01cm, fill=black, below=1cm of deliver] {};
\node (stop2) at (stop) [circle,minimum size=.7cm, line width=.01cm, draw=black] {};
\draw[-{Latex[width=3mm]}] (deliver) -- (stop2);


\draw[-{Latex[width=3mm]}] (copy-src-dest) -- (mf-set);
\draw[-{Latex[width=3mm]}] (mf-set) --node[anchor=east,pos=.0,yshift=-.3cm]{false} (deliver);
\draw[-{Latex[width=3mm]}] (mf-set) --node[anchor=south]{true} (buffer-last);
\draw[-{Latex[width=3mm]}] (buffer-last) |- (deliver);
  \begin{pgfonlayer}{bg}
    \node at (e-bit) [draw,color=white,fill=gray!20,rectangle,yshift=-1cm,minimum width=15cm,minimum height=7cm,label={[xshift=-5.8cm]above:\textsc{Deconcatenation}}] {};

    \node at (append-to-buffer) [draw,color=white,fill=gray!50,rectangle,xshift=3cm,yshift=-.4cm,minimum width=10cm,minimum height=5.5cm,label={[xshift=-3.8cm]above:\textsc{Reassembly}}] {};
  \end{pgfonlayer}
\end{tikzpicture}
%\end{document}

    }
    \caption[Receive Flowchart]{Receive flowchart for proposed modifications}
    \label{fig:receive-flowchart}
\end{figure}
\pagebreak
\subsection{Receiving Process}
The following receive flow is displayed as flowchart in figure~\ref{fig:receive-flowchart}.

When a frame is received it must be validated as specified by the original \gls{MACsec} standard.
Except to this validation is the check if both reserved bits of the \gls{SL} field are set as these are now the \gls{E} and \gls{MF} bit.

If the validation is successful, it is checked whether the \gls{E} bit is set.
If the \gls{E} bit is set, all extension headers are processed and the payload is cut into several \glspl{SDU}.
The length of each one is derived by the extension headers.
The Source- and Destination addresses are prepended to each \gls{SDU}.
If the \gls{E} bit is not set, the payload contains only one \gls{SDU}.

If the fragmentation buffer for the current \gls{SC} contains any data, the first \gls{SDU} must be prepended to this data and the whole, new \gls{SDU} must be delivered.
If the \gls{MF} bit is set, the last \gls{SDU} has to be saved into the buffer for the current \gls{SC}.

Finally, all remaining \glspl{SDU} are delivered.

\subsection{\acrlong{MPDU}}

\begin{figure}
  \centering
  \begin{bytefield}[bitwidth=0.0625\columnwidth]{8}
    \bitheader{1-8} \\
      \wordbox[lrt]{1}{Segment Length} \\
      \bitbox[lrb]{3}{ } & \colorbitbox{lightgray}{4}{} & \bitbox{1}{E}
  \end{bytefield}
  \caption[Extension header of modified \acrshort{MACsec}]{Extension header of modified \gls{MACsec}}
  \label{fig:extension-header}
\end{figure}


The modified \gls{MPDU} is displayed in figure~\ref{fig:concat-sectag}.
The first unused bit after the \acrlong{SL} field is the \gls{MF} bit.
This bit indicates whether the payload is fragmented and the next frame contains the next fragment.
The second unused bit after the \acrlong{SL} field is the \gls{E} bit.
This bit indicates whether an extension header is appended to the \gls{SecTAG}.

The extension header comprises of an 11 bit long \acrfull{SEGL} field and an \gls{E} bit.
These are seperated by 4 unused bits.
The \gls{SEGL} field contains the length of the corresponding \gls{SDU}.
The $n$th extension header corresponds to the $n + 1$th \gls{SDU}.
If the \gls{E} bit of an extension header is set, another extension header follows.
Otherwise, the payload follows.
