%!TEX root=../thesis.tex
\section{Fragmentation in other Protocols}
\label{ch:fragmentation}
In the context of computer networking fragmentation happens when data units are too large to be sent through the network.
If this happens, the datagram is split into several fragments as shown in figure~\ref{fig:fragmentation}.
These are transmitted independently.
The receiver rebuilds the datagram out of the received fragments, this process is called reassembly.

In the following sections, three different protocols are examined regarding fragmentation.
The \gls{IP} is widespread and realizes fragmentation on layer 3.
\gls{WLAN} is a common layer 2 protocol, which implements fragmentation.
Furthermore, \gls{LTE} realizes fragmentation and concatenation on layer 2.
Concatenation is the process of appending several datagrams to one joint datagram.

\begin{figure}
    \centering
    \def\svgwidth{0.5\columnwidth}
    \input{./figures/fragmentation.pdf_tex}
    \caption[Datagram Fragmentation]{A datagram is fragmented into two fragments}
    \label{fig:fragmentation}
\end{figure}
\pagebreak
\subsection{\acrlong{IP} Fragmentation}
\label{sec:ip-fragmentation}
The \gls{IP} fragments internet datagrams into packets matching the path \gls{MTU} as already mentioned in section~\ref{par:layer-3}.
The protocol is specified in~\cite{rfc791}.

Three fields of the \gls{IP} header are relevant for the fragmentation and reassembly process: the \gls{MF} bit, the Fragment-Offset field and the Identification field.

The Identification field is two bytes long and unique for the source-destination pair.
It is determined by the originating protocol that sends internet datagrams and is used to identify the fragmented datagram.
A \gls{MF} bit indicates that the packet is not the last fragment of the fragmented internet datagram.
The fragments position in the datagram is described by a 13 bit long Fragment-Offset field.
It indicates the number of preceding 8 bytes long blocks.
So a Fragment-Offset field with the value of one means that the preceding data is 8 bytes long.
This field allows to reassemble fragments even if they do not arrive in the transmitted order.

Regarding figure~\ref{fig:fragmentation} the first packet---carrying Fragment I in its payload---has the \gls{MF} bit set and the Fragment-Offset is zero.
The second packet---carrying Fragment II in its payload---has the \gls{MF} bit set to zero and the Fragment-Offset is set to the position of this fragment in the datagram.
The Identification field is the same for both packets, as both contain fragments of the same datagram.

The detailed process of the \gls{IP} fragmentation and reassembly process is described in~\cite[section 2.3]{rfc791}

\subsection{\acrshort{IEEE} 802.11 \acrlong{WLAN}}
\gls{WLAN} is a standard for "wireless connectivity [...] within a local area"~\cite[sec. 1.1]{ieeewlan}.
The standard defines several physical layers and a \gls{MAC} layer which contains fragmentation.
It is specified by the \gls{IEEE} in~\cite{ieeewlan}.

Due to the fact that \gls{WLAN} uses a shared medium, collision probability and the probability of interferences is higher compared to Ethernet.
Fragmentation can be used to improve the reliability of \gls{WLAN} \cite[sec. 4.4.3]{tanenbaum2011networks}.

Similar to fragmentation in \gls{IP}, which is described prior to this section, the header of a \gls{WLAN} frame has a \gls{MF} bit.
In addition to that, a Sequence Control field exists, which comprises of a Fragment number field and a Sequence number field.
The Sequence number field identifies the datagram that is fragmented; the Fragment number field identifies the individual fragments.
The first fragment has the Fragment number field set to zero.
It is incremented with every following fragment.
The Sequence number field is the same for every fragment that is part of the same datagram.
It is comparable to the Identification field of \gls{IP}.
\pagebreak
\subsection{\acrlong{LTE}}
\label{sec:lte}
The \gls{LTE} standard is defined by the 3GPP---"a collaboration between telecommunications associations"~\cite{tanenbaum2011networks}---in various specifications~\cite{3gppr8}.
It is a wireless comunication technology used for mobile devices that has fragmentation and concatenation implemented.
This means that---additionally to fragmentation---the payload of several datagrams can be concatenated into one frame.
Fragmentation in the \gls{LTE} standard is named segmentation, but this thesis continues the usage of the term fragmentation to avoid confusion.

\begin{figure}
    \centering
    \def\svgwidth{\columnwidth}
    \input{./figures/rlc.pdf_tex}
    \caption[RLC Fragmentation and Concatenation]{Three \gls{RLC} \glspl{SDU} are transformed into two \gls{RLC} \glspl{PDU}.
    K, K+1 and the first fragment of K+2 are concatenated, K+2 is additionally fragmented.\footnotemark
    }
    \label{fig:RLC}
\end{figure}
\footnotetext{Figure from~\url{http://www.tutorialspoint.com/lte/lte_layers_data_flow.htm}}

The process of fragmentation and concatenation in \gls{LTE} is shown in figure~\ref{fig:RLC}.
The \gls{RLC} layer, which is defined in~\cite{lterlc}, is responsible for fragmentation and concatenation in \gls{LTE}.
This layer lays between the physical layer and the layer which includes \gls{IP}, like layer 2 of the \gls{OSI} model.

The \gls{RLC} layer in \gls{LTE} has different modes but to understand fragmentation and concatenation in \gls{LTE} this distinction is not relevant and, therefore, not deepened here.
In the following the header of the \gls{UM} \gls{PDU} is described to understand how fragmentation and concatenation work in general in \gls{LTE}.

\subsubsection{\acrlong{UMD} \acrshort{PDU}}
The header of an \gls{UMD} \gls{PDU} comprises of two parts: a fixed and an optional extension part.
Furthermore, the \gls{PDU} has one data field which can contain several \glspl{SDU} as shown in figure~\ref{fig:RLC}.

\paragraph{Fixed part}
The fixed part of the header is displayed in figure~\ref{fig:umd-fixed}.
It contains a \gls{FI} field, an \gls{E} bit and a \gls{SN} field.

\begin{figure}[!htbp]
  \centering
  \begin{bytefield}[bitwidth=0.08\columnwidth]{8}
    \bitheader{1-8} \\
    \bitbox{2}{\gls{FI}} & \bitbox{1}{\gls{E}} & \bitbox{5}{\gls{SN}}
  \end{bytefield}
  \caption[Fixed part of an \acrshort{UMD} header]{The fixed part of an \gls{UMD} header.}
  \label{fig:umd-fixed}
\end{figure}

The \gls{FI} field shows, whether the first and last \gls{RLC} \glspl{SDU} are fragmented or not.
The first bit is set, if the first byte of the data field is not equal to the first byte of a \gls{RLC} \gls{SDU}.
The second bit is set, if the last byte of the data field is not equal to the last byte of a \gls{RLC} \gls{SDU}.

For example, the \gls{FI} field of the left \gls{RLC} \gls{PDU} of figure~\ref{fig:RLC} is $01$ because the first \gls{SDU} is not fragmented, but the last.

If the \gls{E} bit of the header is set, an extension part is following. Otherwise the data field follows.

The \gls{SN} field is the identifier of an \gls{UMD} \gls{PDU} and is incremented for the following \gls{PDU}.
This is necessary to reassemble the correct \gls{SDU} fragments together.

\paragraph{Extension part}
If more than one data field is present---this means concatenation is used---one or more extension parts exist in the \gls{PDU} header.
This extension part, displayed in figure~\ref{fig:umd-extension}, contains an \gls{E} bit and a \gls{LI} field.
The \gls{E} bit indicates, whether another extension part or the data field follows.
The \gls{LI} field contains the length of the associated data field.

\begin{figure}[!htbp]
  \centering
  \begin{bytefield}[bitwidth=0.08\columnwidth]{12}
    \bitheader{1-12} \\
    \bitbox{1}{\gls{E}} & \bitbox{11}{\gls{LI}}
  \end{bytefield}
  \caption[Extension part of an \acrshort{UMD} header]{The extension part of an \gls{UMD} header.}
  \label{fig:umd-extension}
\end{figure}

If concatenation happens, for every---except the first---concatenated \gls{SDU} an extension part is added to the header.
This means that for $k$ concatenated \glspl{SDU} the header comprises of one fixed part and $k - 1$ extension parts.
The $n$th extension part is associated with the $n + 1$th \gls{SDU}.
For example, in figure~\ref{fig:RLC} the header of the left \gls{RLC} \gls{PDU} has one fixed part and two extension parts.

The recipient is able to reassemble the fragmented and concatenated \glspl{SDU} with the header information.
With the information of the \gls{SN} field, the order can be reconstructed.
By reading the extension header, the \gls{PDU} can be split up into the different \glspl{SDU}.
The \gls{FI} field describes, wether these \glspl{SDU} are fragments or not, if so the related fragments are concatenated.
