%!TEX root=../thesis.tex
\section{Implementation}
\label{ch:implementation}
In 2016 the \gls{MACsec} kernel driver implemented by \textsc{Sabrina Dubroca} was merged into the linux kernel\footnote{\url{https://git.kernel.org/pub/scm/linux/kernel/git/torvalds/linux.git/commit/?id=c09440f7dcb304002dfced8c0fea289eb25f2da0}}.
The design specified in the previous chapter is implemented based on the kernel version 4.15.6\footnote{\url{https://git.kernel.org/pub/scm/linux/kernel/git/stable/linux-stable-rc.git/commit/?h=v4.15.6&id=1a7aef62b47b00630e62a268d647f54ec93fb38c}}.

In~\cite{dubrocamacsec} \textsc{Dubroca} describes the implementation details of the \gls{MACsec} driver.
\gls{MACsec} is implemented as a virtual network device.
There are two important entry points regarding the proposed changes.
When a new frame is sent via the \gls{MACsec} device, the function \inlinecode{c}{macsec_start_xmit} is executed with the frame as parameter.
When a frame is received by the \gls{MACsec} device, the \inlinecode{c}{rx_handler} \inlinecode{c}{macsec_handle_frame} function is executed.

Frames in the linux kernel are represented by \inlinecode{c}{struct sk_buff}, which is an abbreviation of socket buffer.
This struct contains multiple pointers to the actual data.

The \gls{MACsec} frame header is represented by the \inlinecode{c}{struct macsec_eth_header}.
To reflect the proposed changes of the \gls{MPDU}, it now contains the \acrlong{MF} and \acrlong{E} bits as well as an arbitrary number of extension headers as shown in the implementation excerpt~\ref{lst:macsec-hdr}.

The \gls{MTU} of the \gls{MACsec} device is calculated by subtracting the size of the \gls{SecTAG} and \gls{ICV} from the \gls{MTU} of the underlaying network device.
To reflect the increased \gls{MTU}, a constant---\inlinecode{c}{#define ADDITIONAL_MTU 1468}---is introduced and added to this value.
\begin{figure}
  \lstset{language=C}
  \begin{lstlisting}
  struct macsec_eth_header {
      struct ethhdr eth;
      u8  tci_an;
      u8  short_length:6
          more_fragments:1
          extended:1;
      __be32 packet_number;
      u8 secure_channel_id[8];
  } __packed;
  \end{lstlisting}
  \caption{\acrshort{MACsec} header struct}
  \label{lst:macsec-hdr}
\end{figure}

\subsection{Fragmentation}
The fragmentation is implemented in the transmission handler function \inlinecode{c}{macsec_start_xmit}.
Therefore, the size of the socket buffer is checked.
If its current size added to the additional \gls{MACsec} size is larger than the \gls{MTU} of the underlaying device a fragment has to be created.
Therefore, a second socket buffer is allocated.
The ethernet header is copied and the payload is split up between both frames.
After that, each step is executed on both socket buffers if fragmentation occurs.
To ensure the correct order of both frames, the second socket buffer is transmitted directly after the first socket buffer.

\subsection{Reassembly}
The reassembly takes place after the \gls{MACsec} frame is validated and the \gls{SecTAG} and \gls{ICV} are stripped.
If the \acrfull{MF} bit is set, the payload, its length and the original Ethertype are saved in a double linked list as shown in~\ref{lst:frag_buff}.
As every \acrfull{SC} has its own buffer the network device is still able to receive frames of different \glspl{SC}.

\begin{figure}
  \lstset{language=C}
  \begin{lstlisting}
  struct macsec_fragmentation_buffer {
    sci_t sci;
    struct macsec_fragmentation_buffer *next;
    struct macsec_fragmentation_buffer *prev;
    unsigned char *data;
    unsigned int len;
    unsigned short proto;
  }
  \end{lstlisting}
  \caption{Fragment buffer data structure}
  \label{lst:frag_buff}
\end{figure}

When a frame arrives that does not have the \gls{MF} bit set, it is checked whether data is buffered for this \gls{SC}.
If this is the case, the data of the fragmentation buffer is prepended to the payload of the frame.
Additionally, the Ethertype is set according to the saved one.
After the buffer is cleared, the frame is delivered.

\subsection{Challenges}
While implementing the proposed modifications of \gls{MACsec}, several issues arised.
The three most significant ones are described in the following paragraphs.

\paragraph{Order of frames}
\label{par:frame-order}
The protocol relies on the correct order of frames.
As both endpoints are connected directly, it must be ensured that the network interface retains the order on transmission.
Therefore, the new \gls{MACsec} driver transmits both fragments successively to the network interface.

Each network interface has a configured algorithm which manages the queue.
This group of algorithms is named \glspl{Qdisc}~\cite{hubert2002linux}.
A simple \gls{Qdisc} is \gls{FIFO}---packets are sent in the order they arrive.
There are also \glspl{Qdisc}, which do not retain the order of frames as they arrive, but prefer some with certain properties.
One of them is FQ\_CoDel\footnote{\url{http://man7.org/linux/man-pages/man8/tc-fq_codel.8.html}}, which drops packets with a high queing delay.

The \gls{Qdisc} FQ\_CoDel has been set for the network interface on the used test system.
This led to high retransmission and error rates while using ``iperf3''.

As the virtual device should not change the configuration of the underlaying network interface, it does not set a \gls{Qdisc}.
The only solution that may seem reasonable is to print a warning to kernel log, if a \gls{Qdisc} differs from \gls{FIFO}, as this may decrease the performance of \gls{MACsec} fragmentation significantly.

\paragraph{Concatenation}
The implementation of the concatenation algorithm has several requirements, two of them are the possibility to access and manipulate the queue of the \gls{MACsec} device.
By default the \gls{MACsec} device has no queue but after configuring a \gls{Qdisc}---which is described in the previous section---it was possible to access it.
When concatenation of several \glspl{SDU} happens, it may occur that the last \gls{SDU} must additionally be fragmented.
If this happens, it must be guaranteed that the last fragment of this \gls{SDU} is sent in the next \gls{MACsec} frame.
Thus, this fragment must be the next one processed by the \gls{MACsec} device and the queue must be manipulated to achieve that.
After some investigation, it was possible to dequeue and enqueue \glspl{SDU}, but not to manipulate the order of the queue.
For time reasons, it was not possible to implement this manipulation and hence the current implementation does not support concatentation.

\paragraph{Fragmentation of TCP packets}
If the socket buffer has to be fragmented, it is split into two parts.
Therefore, the original buffer is trimmed and the remaining payload is moved to another position, where the fragment points to.

To test the implementation the command line tools \inlinecode{bash}{ping}---which sends ICMP packets---and  \inlinecode{bash}{iperf3}---which sends \gls{TCP} packets---where used.
In the beginning of each test everything worked but when \inlinecode{bash}{iperf3} was used after a varying amount of time the kernel crashed.

After many debugging attempts, the reliability of \gls{TCP} turned out to be a problem.
\gls{TCP} retransmits a packet when no acknowledge is received.
Therefore, the kernel keeps a pointer of the socket buffer, so on retransmission the frame must not be recreated.
If the socket buffer was fragmented by \gls{MACsec} the kernel expected a longer one which led to a kernel panic.

To solve this problem, the implementation now checks, whether a reference to the socket buffer is still hold by anyone.
If this is the case the socket buffer and its data is cloned, so if \gls{TCP} now retransmits a packet it is not modified.
This is recommended by the documentation\footnote{\url{https://www.fsl.cs.sunysb.edu/kernel-api/re501.html}}.
