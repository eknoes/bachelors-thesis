%!TEX root=../thesis.tex
\chapter{Evaluation}
\label{ch:eva}
% Intel(R) Core(TM) i5-4590 CPU @ 3.30GHz
% Intel® Ethernet Connection I217-LM
% 16GB RAM
% CAT 5E Gigabit Ethernet

% Erst Ziele, warum? Messwerte -> Setup -> Ergebnisse -> Auswertung -> Bewertung
The problem of a decreased \gls{MTU} when using \gls{MACsec} is solved with the proposed modifications of chapter~\ref{ch:design}, which were implemented as described in section~\ref{ch:implementation}.

To assess, if these modifications lead to significant performance changes, several measurments were made.
These are compared with the solution of using jumbo frames, which is also a possibility to mitigate the problem of a lower \gls{MTU}.
This solution is considered to be optimal but not applicable in the given scenario.

As the modified driver is sending two frames instead of one considering a large frame, \acrfull{RTT} might be affected.
Furthermore, an additional \gls{MACsec} header must be sent, which might affect the bandwidth.
As a third measurement the CPU usage during the bandwidth test is recorded, to see if the fragmentation and reassembly has a significant impact on it.

For the evaluation two computers are directly connected to each other with a Cat5e network cable.
Table~\ref{tab:eva-hardware} contains the hardware specifications of both machines.
\begin{table}[h]
  \centering
  \begin{tabular}{l  r}
    Processor & Intel i5-4590 3.3GHz\\
    RAM & 16GB\\
    Network interface & Fast Ethernet \& Gigabit Ethernet\\
    Operating System & Arch Linux\footnotemark with linux kernel 4.15.7\\
  \end{tabular}
  \caption{Specifications of the machines used for the evaluation}
  \label{tab:eva-hardware}
\end{table}
\footnotetext{\url{https://www.archlinux.org/}}

\section{Experiment}
Evaluated is the \gls{RTT} and bandwidth.
For the measurement of the \gls{RTT} the command line tool ``ping'' is used.
The bandwidth is measured with ``iperf3''\footnote{\url{http://software.es.net/iperf/}}.

The test cases are:
\begin{itemize}
  \item Ethernet with \gls{MTU} of 1468 bytes, 1500 bytes and 2936 bytes
  \item \gls{MACsec} with \gls{MTU} of 1468 bytes, 1500 bytes and 2936 bytes by enabling Jumbo Frames
  \item Modified \gls{MACsec} with \gls{MTU} of 1468 bytes, 1500 bytes and 2936 bytes
\end{itemize}

The \gls{MTU} of 1468 bytes is chosen because it is the standard \gls{MACsec} \gls{MTU}.
With this \gls{MTU} the modified \gls{MACsec} does not do any fragmentation.
The \gls{MTU} of 1500 bytes is chosen because it is the standard ethernet \gls{MTU}.
And with an \gls{MTU} of 2936 bytes the modified \gls{MACsec} version sends two full frames, instead of one full and one small frame as in the setting with an \gls{MTU} of 1500 bytes.

The measurement is automated with a shell script, which runs on one of both machines.
In the following, this machine is referred to as sender and the other one as receipient.

In the beginning, the sender sets the configuration according to the current test case.
After that, it configures the receipient and starts an iperf3 server via ssh.
A single bandwidth measurement performs a 10~second long iperf3 test, which builds up a \gls{TCP} connection and measures the average bandwidth.
This test is repeated 1000 times for each test case, except the Ethernet case with a \gls{MTU} of 2936 bytes where it is repeated 100 times.
The ethernet test case is interesting for comparability but plays a minor role for the assessment of the \gls{MACsec} modifications.
For the \gls{RTT} measurment an adaptive ping is started, which sends and receives 50000 \gls{ICMP} messages.
Usually, ping sends one echo request every second but with the adaptive ping it sends the next request as soon as the last was acknowledged.

\section{Measurements}
\subsection{\acrlong{RTT}}
\begin{figure}
    \centering
    \def\svgwidth{\columnwidth}
    \input{./figures/rtt-boxplot.pdf_tex}
    \caption[\acrlong{RTT} experiment]{\gls{RTT} measurements for the different test cases.\\$n=50000$}
    \label{fig:rtt-test}
\end{figure}

The results of the \gls{RTT} are displayed in figure~\ref{fig:rtt-test} and table~\ref{tab:rtt-data}.
It is noticeable that two test cases which used jumbo frames---Ethernet and \gls{MACsec} with an \gls{MTU} of 2936 bytes---have a significant higher \gls{RTT} (0.44ms to 0.52ms) and higher variance, compared with the remaining test cases (0.14ms to 0.17ms).

The reasons for this are not further researched in this thesis but there are several publications which detect a high latency when using jumbo frames~\cite{chelseajumbo}.

To make the results of the other test cases more visible, both described cases are removed in figure~\ref{fig:rtt-test-zoom}.
\begin{figure}
    \centering
    \def\svgwidth{\columnwidth}
    \input{./figures/rtt-boxplot-zoom.pdf_tex}
    \caption[\acrlong{RTT} experiment (clipped)]{Clipped \gls{RTT} measurements for the different test cases, full results visible in figure~\ref{fig:rtt-test}.\\$n=50000$}
    \label{fig:rtt-test-zoom}
\end{figure}

The \gls{RTT} of Ethernet with an \gls{MTU} of 1468 bytes (0.139ms) respectively 1500 bytes (0.14ms) is lower than the \gls{RTT} of the test cases where \gls{MACsec} is used.
It can be assumed that this is a result of the cryptographic operations when using \gls{MACsec}.

It is noticeable that the modified \gls{MACsec} (0.155ms) has a similar \gls{RTT} as the unmodified \gls{MACsec} (0.154ms) with a frame size of 1468 bytes.
Here, no fragmentation happens.
When the modified \gls{MACsec} does fragmentation with a frame size of 1500 bytes, the \gls{RTT} increases to 0.166ms.
Compared to \gls{MACsec} with jumbo frames and a \gls{MTU} of 1500 bytes (0.155ms) this is an increase of 7.1\%.
It is assumed that this is the result of having to send two frames instead of one.
The modified \gls{MACsec} with an \gls{MTU} of 2936 bytes has an increased \gls{RTT} (0.174ms), which might be the result of a bigger frame which has to be sent.
But compared to \gls{MACsec} with jumbo frames and an \gls{MTU} of 2936 bytes it is significant faster.

\subsection{Bandwidth}
\label{sec:bandwidth-eva}
\begin{figure}
    \centering
    \def\svgwidth{\columnwidth}
    \input{./figures/bandwidth-bar.pdf_tex}
    \caption[Bandwidth experiment bar chart]{Bandwidth measurements for the different test cases.\\$n=1000$, except for Ethernet with MTU 2936 $n=100$}
    \label{fig:bandwidth-test}
\end{figure}
The results of the bandwidth test are shown in figure~\ref{fig:bandwidth-test} and table~\ref{tab:bandwidth-data}.

The data shows that, independent of the test case, the variance is small.
All test cases where no \gls{MACsec} is used reach a higher bandwidth compared with the other test cases which had the same \gls{MTU}.
This decrease when using \gls{MACsec} might be the result of the additional 32 bytes long header, which has to be sent in each frame.

\gls{MACsec} (907.4 Mbit/s) and the modified \gls{MACsec} (907.38 Mbit/s) reach a similar bandwidth when using a frame size of 1468 bytes.
When the modified \gls{MACsec} does fragmentation, the bandwidth decreases compared to \gls{MACsec} with the same \gls{MTU} of 1500 bytes respectively 2936 bytes.
In the fragmentation cases a second Ethernet and \gls{MACsec} header have to be sent which lead to this decrease.
The modified \gls{MACsec} reaches 95\% respectively 97\% of the bandwidth that \gls{MACsec} with the same \gls{MTU} creates.

\subsection{CPU Usage}
The results of the CPU usage during the bandwidth test are shown in figure~\ref{fig:cpu-test} and table~\ref{tab:cpu-data}.
As the data is generated during the bandwidth test, the CPU usage also depends on the size of this data.

During the tests without \gls{MACsec}, the CPU usage is significantly higher (5.55\% to 5.16\%) than during the tests where \gls{MACsec} is applied (0.96\% to 2.29\%).
Furthermore, it is noticeable that the CPU usage during the tests with the modified \gls{MACsec} is in the same order of magnitude as during the use of \gls{MACsec}.

There is at least no obvious reason, why during the tests without the usage of \gls{MACsec} the CPU was used significantly more.
These were the cases, where more data for the bandwidth test was generated due to a better performance as evaluated in section~\ref{sec:bandwidth-eva}.
For investigation of this reason, a normalization of the CPU usage result was made, which is displayed in figure~\ref{fig:cpu-test-norm}.
But even here the results are significantly worse.
The results of the CPU usage measurement need more investigation.

\begin{figure}
    \centering
    \def\svgwidth{\columnwidth}
    \input{./figures/cpu-usage-boxplot.pdf_tex}
    \caption[CPU Usage experiment]{CPU usage during the different test cases regarding the bandwidth test.\\$n=1000$, except for Ethernet with MTU 2936 $n=100$}
    \label{fig:cpu-test}
\end{figure}
\begin{figure}
    \centering
    \def\svgwidth{\columnwidth}
    \input{./figures/cpu-bandwidth-boxplot.pdf_tex}
    \caption[CPU Usage normalized to sent megabytes]{Sent megabytes divided by CPU usage in percent during bandwidth test.\\$n=1000$, except for Ethernet with MTU 2936 $n=100$}
    \label{fig:cpu-test-norm}
\end{figure}

\section{Security Evaluation}
This thesis assumes that \gls{MACsec} is secure.
It provides integrity, confidentiality, data origin authenticity, and protection against replay attacks.

The proposed modifications alter the \gls{SecTAG}.
Two reserved bits are replaced by the \acrlong{MF} and \acrlong{E} bit.
Furthermore, additional extension headers may be appended to the \gls{SecTAG}.

These modifications---which are explained in further detail in chapter~\ref{ch:proposal}---are evaluated regarding security in the following paragraphs.

\subsection{Confidentiality}
\gls{MACsec} provides confidentiality by encrypting the secure payload.
This does not include the \gls{SecTAG}.

If fragmentation is applied, the \gls{SDU} is split and each part is the secure payload of one \gls{MPDU}.
If concatenation is applied, several \glspl{SDU} build the secure payload of one \gls{MPDU}.
As the secure payload is constructed before \gls{MACsec} is applied, it still provides the same confidentiality regarding the encryption of the transmitted data.
The encryption process as well as the used algorithms or keys are not modified.

The \gls{SecTAG} is extended with information of whether a fragment follows and the length of each concatenated \gls{SDU}.
The number and corresponding length of transmitted \glspl{SDU} is obtainable for an attacker, when the unmodified \gls{MACsec} is used.
Hence, no additional information is disclosed.

Therefore, confidentiality is not weakened by the proposed modifications.

\subsection{Integrity}
\gls{MACsec} provides integrity by building a message authentication code named \gls{ICV}.
The \gls{ICV} includes the source and destination adresses, the \gls{SecTAG} and the secure payload.

Fragmented as well as concatenated \glspl{SDU} are always contained in the secure payload of an \gls{MPDU} and, therefore, integrity protected by the \gls{ICV}.
The modifications of the \gls{SecTAG} are also protected by the \gls{ICV}.
The process defined in chapter~\ref{ch:proposal} specifies that on transmit the modifications and the secure payload are built before calculating the \gls{ICV}.
Furthermore, on receive the \gls{ICV} is verified before the data is processed.
Thus, an outside attacker has no possibility to modify data undetectable.
This is visible in the flowcharts describing the transmission (figure~\ref{fig:fragmentation-flowchart}) and receive (figure~\ref{fig:receive-flowchart}) process.

Therefore, the integrity is not weakened by the proposed modifications.
The data origin authenticity is guaranteed by integrity and authentication through the \acrlong{KaY}.
As integrity is not weakened and the \acrlong{KaY} is not part of \gls{MACsec} standard data origin authenticity is still ensured.

\subsection{Availability}
The proposed modifications of \gls{MACsec} specify that received data is verified by \gls{MACsec} before the process of reassembly and deconcatenation occurs.
Therefore, outside attackers cannot exploit the introduced behavior, as it is assumed that \gls{MACsec} is secure.
On transmission and receive, the proposed modifications perform similar to \gls{MACsec} as shown in the previous sections.
Inside attackers which participate legitimately in the communication can exploit the introduced behavior regarding availability, for example by sending frames which have the \gls{MF} bit set.
This leads to a reassembly of not associated fragments and therefore illegable data is delivered.
But as \gls{MACsec} does not prevent insider attacks, this is not considered.

Therefore, the availability regarding outside attackers is not weakened by the proposed modifications.

\subsection{Replay protection}
\gls{MACsec} provides replay protection by using incrementing packet numbers.
Packets with a number which was already received are dropped\footnote{The exact behavior is configurable}.
This behavior is not changed by the modifications, the check of the packet number is done before the modifications are applied.
Furthermore, the field containing and the build of the packet number is not changed.

Therefore, the replay protection of \gls{MACsec} is not weakened by the proposed modifications.
\pagebreak
\section{Results}
The results of the performance evaluation and security evaluation are satisfying.

The \gls{RTT} results are as expected if fragmentation is applied the \gls{RTT} just slightly increases.
Additionally, the results of the bandwidth test shows that the modified \gls{MACsec} still reaches 95\% of the results of \gls{MACsec} with jumbo frames.
The measurements of the CPU usage are not completely comprehensible but the results of using modified \gls{MACsec} are in the same order of magnitude as unmodified \gls{MACsec}.
The security evaluation shows that the proposed modifications maintain the security level of \gls{MACsec}.

The proposed modifications appear to successfully solve the given problem without decreasing the performance significantly, as demonstrated by the prior evaluation.
A working implementation as a linux kernel module is provided.
This outcome leads to the conclusion that functionality of the proposed solution is given.
